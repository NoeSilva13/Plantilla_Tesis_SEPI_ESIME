%------------------------------------------
% Contiene la �ltima p�gina
%-----------------------------------------
% Ponemos el marcador en el PDF al nivel adecuado, dependiendo
% de su hubo partes en el documento o no (si las hay, queremos
% que aparezca "al mismo nivel" que las partes.
\ifpdf
\ifx\tienePartesTeXiS\undefined
   \pdfbookmark[0]{Fin}{fin}
\else
   \pdfbookmark[-1]{Fin}{fin}
\fi
\fi
\thispagestyle{empty}\mbox{}

\vspace*{4cm}

\small

\hfill \emph{Su libertad M�xico crea,}

\hfill \emph{surge la Patria nace la luz;}

\hfill \emph{nos convoca tu voz, Polit�cnico,}

\hfill \emph{nos conduce tu amor, juventud.}

\hfill 

\hfill \emph{Coro del Himno del Instituto Polit�cnico Nacional} 
\newpage
\thispagestyle{empty}\mbox{}

\newpage

% Variable local para emacs, para  que encuentre el fichero maestro de
% compilaci�n y funcionen mejor algunas teclas r�pidas de AucTeX

%%%
%%% Local Variables:
%%% mode: latex
%%% TeX-master: "../Tesis.tex"
%%% End:
