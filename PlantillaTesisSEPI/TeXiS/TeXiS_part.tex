%---------------------------------------------------------------------
% Fichero que define los comandos necesarios para crear distintas
% partes del documento.
%---------------------------------------------------------------------
% T�tulo de la parte
\def\partTitleVal{}
\newcommand{\partTitle}[1]{
\def\partTitleVal{#1}
}
% Descripci�n de la parte
\def\partDescVal{}
\newcommand{\partDesc}[1]{
\def\partDescVal{#1}
}
\def\partBackTextVal{}
\newcommand{\partBackText}[1]{
	\def\partBackTextVal{#1}
}
%%%
% Configuraci�n terminada
%%%
%%%
%% COMANDOS NECESARIOS PARA CREAR LA PARTE
%%%
\newcommand{\makepart}{\part{\partTitleVal}\partDesc{}\def\tienePartesTeXiS{}}
\newcommand{\makespart}{\part*{\partTitleVal}\partDesc{}\partBackText{}\def\tienePartesTeXiS{}}



% El \makepart se basa en  el comando LaTeX \part{...} que redefinimos
% aqu� (cambiando la  definici�n de book) para permitir  el uso de los
% comandos anteriores.  En realidad el usuario  podr�a seguir haciendo
% uso del comando de LaTeX \part, pero podr�a a�adir las descripciones
% en la propia p�gina, ni en la p�gina trasera.

\makeatletter
\renewcommand\part{%
  \if@openright
    \cleardoublepage
  \else
    \clearpage
  \fi
  \thispagestyle{empty}% Cambiado; antes era plain pero
% no queremos n�mero de p�gina...
  \if@twocolumn
    \onecolumn
    \@tempswatrue
  \else
    \@tempswafalse
  \fi
  \null\vspace*{2cm}% Cambiado; en vez de \vfil, un espacio fijo de
                    % dos cent�metros, para que el t�tulo no cambie de
                    % posici�n dependiendo de la longitud de la descripci�n.
  \secdef\@part\@spart}

% Versi�n "sin estrella" del comando part. Se mete en el toc y se pone
% "Parte I"; despu�s viene el t�tulo y descripci�n.
\def\@part[#1]#2{%
    \ifnum \c@secnumdepth >-2\relax
      \refstepcounter{part}%
      \addcontentsline{toc}{part}{\thepart\hspace{1em}#1}%
    \else
      \addcontentsline{toc}{part}{#1}%
    \fi
    \markboth{}{}%
    {\centering
     \interlinepenalty \@M
     \normalfont
     \ifnum \c@secnumdepth >-2\relax
       \huge\bfseries \partname\nobreakspace\thepart
       \par
       \vskip 20\p@
     \fi
     \Huge \bfseries #2\par}
    % Si hay descripci�n de la parte, la ponemos despu�s de 2 cent�metros
    {\vspace*{2cm}\noindent\partDescVal}%
    \@endpart}
% Versi�n "con estrella" del comando part. Se pone el t�tulo y la
% descripci�n.
\def\@spart#1{%
    {\centering
     \interlinepenalty \@M
     \normalfont
     \Huge \bfseries #1\par%
    % Si hay descripci�n de la parte, la ponemos despu�s de 2 cent�metros
    {\vspace*{2cm}\noindent\partDescVal}%
}
    \@endpart}
\def\@endpart{\vfil\newpage
	\if@twoside
	\if@openright
	\null
	\thispagestyle{empty}%
	{\partBackTextVal}
	\newpage
	\fi
	\fi
	\if@tempswa
	\twocolumn
	\fi}
\makeatother